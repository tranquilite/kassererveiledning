\section{FAQ - Ofte Stilte Spørsmål}
\let\thefootnote\relax\footnote{Jaok, så det burde være Ofte \emph{Stilte} Spørsmål, men du kan serr ta deg en bolle}

\subsubsection*{Hvor fører jeg mosjonspengene?}
For mosjonstimer føres det først en fordring mot mosjonsutvalget, som dateres dagen dere får tildelt mosjonstimen.
\postering{1570 - Fordring Mosjonsutvalget}{3845 - Mosjonsinntekter}
Når kravet gjøres opp og penger kommer inn på konto på slutten av semesteret, føres
\postering{1920 - Bank}{1570 - Fordring Mosjonsutvalget}

Er fordringen \emph{ikke} ført, vil ikke kravet stå, og mosjonspengene utbetales ikke!

\subsubsection*{Hva gjør jeg med valutagevinst/-tap?}
Valutagevinst, type \emph{oj, vekslingskursen går i vår favør}, føres som {\bfseries 3999 - diverse inntekter}. \newline
Valutatap føres, avhengig av hvordan gruppa vil takle det, enten som fordring til medlemmet (hvis dere absolutt vil ha pengene), eller som {\bfseries 7830 - tap på fordring}.

\subsubsection*{Hvordan fører jeg depositum}
Først, hvaslags depositum? Har gruppa betalt depositum til \emph{noen} for f.eks leie? Dette er typisk en fordring dere har hos utleier inntil depositumet gjøres opp, og føres 
\postering{1920 Bank}{1396 Depositum}

Hvis det derimot er den type depositum der medlemmene har betalt inn til gruppa, f.eks for reisekostnader dere ikke var \emph{helt} sikre på størrelsen på, men samlet inn cash fra medlemmene for å ha likviditet for reisen, så blir det mer type:
\postering{2980 Gjeld medlemmer}{1920 Bank}
og etter fullendt reise når dere vet hva kostnaden blir
\postering{3150 Egenandel Reise}{2980 Gjeld medlemmer}
etterfulgt av
\postering{1920 Bank}{4031 Reise Buss/Bil/Tog}
det siste også helst i samme bilag, og naturligvis ev. også
\postering{2980 Gjeld medlemmer}{1920 Bank}
hvis det skulle være betalt inn for mye (i forhold til hva egenandelen endte opp på).
