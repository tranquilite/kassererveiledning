\hypertarget{ff_egenkap}{}
\subsubsection*{Egenkapital}
Du kan tenke på egenkapital som en spesiell type gjeld som du har til deg selv.
Egenkapital er definert som den \emph{andelen av kapitalen som tilfaller eierne av foretaket}. I vårt tilfelle er NTNUI en selveiende enhet, så en hel haug unødvendige implikasjoner faller bort.\\
Egenkapital deles inn i \emph{fri} og \emph{bundet}. Den frie egenkapitalen er den vi oftest leker med (hos oss ført mot {\bfseries 2050}), og sammen med ev. gjeld, så er det den som effektivt finansierer eiendelene/driftsmidlene. Den bundne egenkapitalen er EK (orker ikke skrive \emph{egenkapital} hele tiden..) vi selv innrømmer å låse vekk fra fri disposisjon. I NTNUI er dette typisk avsetninger for langsiktig sparing eller - i HS-regnskapet - medlemskontingenten for NTNUI.